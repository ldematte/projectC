\documentclass[a4paper]{article}
\usepackage[a4paper,margin=2cm]{geometry}
\usepackage[T1]{fontenc}
\usepackage{color}
\usepackage{alltt}
\usepackage{times}
\newcommand{\scitea}[1]{\noindent{\ttfamily{\textcolor[rgb]{0.5, 0.5, 0.5}{#1}}}}
\newcommand{\scitec}[1]{\noindent{\ttfamily{\textcolor[rgb]{0.0, 0.5, 0.0}{#1}}}}
\newcommand{\scitee}[1]{\noindent{\ttfamily{\textcolor[rgb]{0.0, 0.5, 0.5}{#1}}}}
\newcommand{\scitef}[1]{\noindent{\ttfamily{\textbf{\textcolor[rgb]{0.0, 0.0, 0.5}{#1}}}}}
\newcommand{\scitek}[1]{\noindent{\ttfamily{\textbf{\textcolor[rgb]{0.0, 0.0, 0.0}{#1}}}}}
\newcommand{\scitel}[1]{\noindent{\ttfamily{\textcolor[rgb]{0.0, 0.0, 0.0}{#1}}}}
\newcommand{\sciteib}[1]{\noindent{\ttfamily{\textcolor[rgb]{0.5, 0.3, 0.1}{\colorbox[rgb]{0.9, 0.9, 1.0}{#1}}}}}
\begin{document}

Source File: 

\noindent
\tiny{
\scitef{class}
\scitea{ }
\scitel{A}
\scitek{;}
\scitea{} \\
\scitef{int}
\scitea{ }
\scitel{B}
\scitea{ }
\scitek{=}
\scitea{ }
\scitee{10}
\scitek{;}
\scitea{} \\
\scitea{} \\
\scitel{A}
\scitea{ }
\scitel{f}
\scitek{(}
\scitel{B}
\scitek{);}
\scitea{ }
\scitec{//dichiarazione di variabile f oggetto della classe A} \\
\scitea{{\hspace*{1em}}{\hspace*{1em}}{\hspace*{1em}}{\hspace*{1em}}{\hspace*{1em}}{\hspace*{1em}}{\hspace*{1em}} }
\scitec{//con un costruttore che richiede un intero} \\
\scitea{{\hspace*{1em}}{\hspace*{1em}}{\hspace*{1em}}{\hspace*{1em}}{\hspace*{1em}}{\hspace*{1em}}{\hspace*{1em}} }
\scitec{//B: IDENTIFIER}
} %end tiny

\end{document}
